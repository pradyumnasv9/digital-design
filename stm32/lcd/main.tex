We show how to interface the $16 \times 2$ HD44780-controlled LCD using STM32F103C8T6.
\begin{enumerate}[label=\arabic*.,ref=\theenumi]
%\begin{enumerate}[label=\thesubsection.\arabic*.,ref=\thesubsection.\theenumi]
\item Make connections as shown in Table \ref{table:lcd_pins}.
\begin{table}[!h]
\input{./stm32/lcd/figs/pins.tex}
\caption{Pin Connections}
\label{table:lcd_pins}
\end{table}
\item Execute the following program
\begin{lstlisting}
lcd/codes/lcd_example.c
\end{lstlisting}
The following problems explain how to display the string \textbf{0} on the screen using
the above code.
\item Write the ASCII code for the 0 character.
\\
\solution The code for 0 is \textbf{48 =0b00110000 = 0x30}.
\\
\item How is 0 written by the STM32 to the LCD controller.
\\
\solution For the number 0, the upper nibble 0011 is first written followed by the lower nibble 0000.
This is done by
\begin{lstlisting}
void SendByte (byte data)
{

 SendNibble(data >> 4); // send upper 4 bits 0011
 SendNibble(data); // send lower 4 bits 0000
}
\end{lstlisting}
\item How is the nibble 0011 written to the LCD.
\\
\solution This is done by the following function where data = 0011.
\begin{lstlisting}
void SendNibble(byte data)
{
GPIOA->BRR = ~(data << 2) & 0b00111100;
GPIOA->BSRR = (data << 2) & 0b00111100;
 
 PulseEnableLine(); // clock 4 bits into controller
}
\end{lstlisting}
The expression
\begin{align}
GPIOA->BSRR = (data << 2) \& 0b00111100 
= 0b00001100.  
\end{align}
This ensures that 11 is written to the pins A2-A3.  Note that $<<$ indicates 2 left shifts. Similarly, 
\begin{equation}
GPIOA->BRR = ~(data << 2) \& 0b00111100
\end{equation}
\item ensures that 00 is written to the pins A4-A5. PulseEnableLine() provides a clock pulse used
to write the nibble 0011 to the LCD.
\item Which pins of the STM32 are used for what purpose?
\\
\solution The A2-A5 pins of the STM32 are used for pushing the upper/lower data nibble to the DB4-DB7 pins
of the LCD using the BRR and BSRR registers.   The A0-A1 pins are used for Register Select and EN for the LCD.
\item What is Register Select?
\\
\solution Register Select = 0 implies that LCD configuration commands are being written. For example, cursor on/off, clearing
display, number of lines, etc... Register Select = 1 implies that
characters are being writen to the LCD.
\item Develop an arithmetic calculator using the STM32 along with the LCD.
\end{enumerate}
