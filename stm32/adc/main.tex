We show how to use the Internal Temperature Sensor to measure the CPU temperature through the ADC.
\begin{enumerate}[label=\arabic*.,ref=\theenumi]
%\begin{enumerate}[label=\thesubsection.\arabic*.,ref=\thesubsection.\theenumi]
%\begin{enumerate}[label={\small\thesubsection.\arabic*.},ref=\thesubsection.\theenumi]
\item Make connections as shown in Table \ref{table:lcd_pins}.
\item Execute the following program
\begin{lstlisting}
stm32/adc/codes/internal_temp.c
\end{lstlisting}
\item What do you observe?
\\
\solution You should observe a number between 1750-1760. This is the output of the internal temperature
sensor, captured in $ADC1->DR$.
\item Find an expression for $V_{SENSE}$
\\
\solution
\begin{equation}
V_{SENSE} = 3.3 \times \frac{ADC1->DR}{4095}
\end{equation}
%
\item Obtain the formula for finding the temperature of the STM32 and list the values of the various parameters.
\\
\solution The desired formula is
\begin{equation}
T = (V_{25}-V_{SENSE})/AvgSlope + 25
\end{equation}
where the typical values of the above parameters are 
\input{./stm32/adc/figs/sensor.tex}
%
\item What is the default ADC frequency?
\\
\solution The ADC operates at 14 MHz by default and is indpendent of the processor frequency (8 MHz in this case). It can, however be synchronized with
the processor clock for some real time applications.
\item Explain the significance of the following instruction
\begin{lstlisting}
ADC1->SMPR1 |= ADC_SMPR1_SMP16;
\end{lstlisting}
\solution Through this command, $ADC1->SMPR1$ = 0x001C0000 where the SMPR1 register is shown in Fig. \ref{fig:smpr1}.  Note that
this makes SMP16 = 111 which means that channel 16 sample time = 239.5 cycles.  Channel 16 is reserved for the internal temperature 
sensor and is connected to ADC1.
%
\begin{figure}
\centering
%\includegraphics[width=\columnwidth]{./stm32/adc/figs/smpr1.eps}
\includegraphics[width=\columnwidth]{./stm32/adc/figs/smpr1}
\caption{SMPR1}
\label{fig:smpr1}
\end{figure}
%
\item What is the sampling time?
\\
\solution Since the sample time is 239.5 cycles and the ADC frequency is 14 MHz, 
\begin{equation}
T_{s} = 239.5 \times \frac{1}{14} \mu s = 17.1 \mu s 
\end{equation}
\item Explain the following instruction.
\begin{lstlisting}
ADC1->SQR3 |= ADC_SQR3_SQ1_4;
\end{lstlisting}

\solution ADC\_SQR3\_SQ1\_4 = 0x00000010.  This implies that SQ1=0b10000 in
the ADC regular sequence register 3 ($ADC1->SQR3$)shown in Fig. \ref{fig:sqr3}. Since SQ1=16,
this means that the ADC input in channel 16 will be the first in the queue
for conversion. 
%
\begin{figure}
\centering
\includegraphics[width=\columnwidth]{./stm32/adc/figs/sqr3}
\caption{SQR3}
\label{fig:sqr3}
\end{figure}
%
The ADC is capable of converting analog 16 inputs one after the other. 
The inputs are called {\em channels} and the sequence number corresponding 
to the channel
is decided
according to the 5 bit entry in SQ. 
\subsection{Measuring an Unkown Resistance}
\item Configure SQR3 so that the 9th channel for ADC1 is 2nd in sequence.
\\
\solution This implies that SQ2=1001.  Thus,
\begin{equation}
ADC1->SQR3 = 0x000000120
\end{equation}
\item List the various pin numbers corresponding to the different channels of the ADC.
\\
\solution See Fig. \ref{table:pin_channel}
\input{./stm32/adc/figs/pin_channel.tex}
\iffalse
\begin{table}[!ht]
\footnotesize
\input{stm32/adc/figs/pin_channel.tex}
\caption{ADC Analog Input Pins}
\label{table:pin_channel}
\end{table}
\fi
\item Use the 9th channel of  ADC1 in SQ2 to measure 3.3V. 
\\
\solution Connect PB1 (Fig. \ref{table:pin_channel}) to 3.3 V of the STM 32 and execute the following
code.
\begin{lstlisting}
stm32/adc/codes/3_3V.c
\end{lstlisting}
%\section{Project}
\item Measure an unkown resistance using the STM32 and display the result on the LCD.
\item Display the output of the internal temperature sensor as well the unknown resistance on the LCD.
\end{enumerate}

