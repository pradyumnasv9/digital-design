Here we show how to program the ESP32 on the Vaman using the Arduino framework.
%\subsubsection{Connections}
\begin{enumerate}[label=\arabic*.,ref=\theenumi]
\item Make sure that Vaman board does not power any devices.  
\item Make connections as shown in \autoref{tab:arduino-uart}.
	%and \autoref{fig:vaman/uart/1}. 
\item The Vaman pin diagram is available in \autoref{fig:vaman-pin_sheet}.
\begin{figure}
\centering
\includegraphics[width=\columnwidth]{vaman/esp32/vaman-esp32/lcd/figs/pin_sheet.png}
	\caption{Vaman pins.  Right side reperesents ESP32 and left side M4-FPGA (Pygmy).}
\label{fig:vaman-pin_sheet}
\end{figure}
\begin{table}[!ht]
	\centering
\input{vaman/esp32/vaman-esp32/arduino-uart/tables/rpi-vaman-uart.tex}
\caption{}
\label{tab:arduino-uart}
\end{table}
%
\item Connect the UART to raspberry pi through USB.  
\item Connect the pins between Vaman-ESP32 and Vaman-PYGMY as per \autoref{tab:vaman/esp32/setup/onboard}
\begin{table}[h]
\centering
\input{./vaman/esp32/setup/tables/onboard.tex}
\caption{}
\label{tab:vaman/esp32/setup/onboard}
\end{table}
\item On termux on your phone, 
\begin{lstlisting}
cd vaman/esp32/codes/ide/blink
pio run
\end{lstlisting}
\item Transfer the ini and bin files to the rpi 
\begin{lstlisting}
scp platformio.ini pi@192.168.50.252:./hi/platformio.ini

scp .pio/build/esp32doit-devkit-v1/firmware.bin pi@192.168.50.252:./hi/.pio/build/esp32doit-devkit-v1/firmware.bin
\end{lstlisting}
\item On rpi,
\begin{lstlisting}
cd /home/pi/hi
pio run -t nobuild -t upload
\end{lstlisting}
You should see the blue led blinking.
\item Now disconnect pin 2 from pin 18 and connect to pygmy GPIO pin 21.  
\item Repeat the above exercise using 
	GPIO pin 22.
\item On your phone, open 
\begin{lstlisting}
src/main.cpp 
\end{lstlisting}
and change the delay to 
\begin{lstlisting}
delay(100);
\end{lstlisting}
and execute the code by following the steps above.
\item Flash the following code. 
\begin{lstlisting}
vaman/esp32/codes/ide/ota/setup
\end{lstlisting}
		after entering your wifi username and password (in quotes below)
\begin{lstlisting}
#define STASSID "..." // Add your network credentials
#define STAPSK  "..."
\end{lstlisting}
in src/main.cpp file
\item You should be able to find the ip address of your vaman-esp using 
\begin{lstlisting}
ifconfig
nmap -sn 192.168.231.1/24
\end{lstlisting}
where your computer's ip address is the output of ifconfig and given by 192.168.231.x
\item Assuming that the username is gvv and password is abcd, flash the following code wirelessly
\begin{lstlisting}
vaman/esp32/codes/ide/ota/blink
\end{lstlisting}
through 
\begin{lstlisting}
pio run 
pio run -t nobuild -t upload --upload-port 192.168.231.245
\end{lstlisting}
where you may replace the above ip address with the ip address of your vaman-esp.
\item Flash the following code OTA
\begin{lstlisting}
vaman/esp32/codes/ide/ota/blinkt
\end{lstlisting}
You should see the onboard LEDs blinking.
\item Change the blink duration to 100 ms.
\item Execute the code in 
\begin{lstlisting}
vaman/esp32/codes/ide/ota/sevenseg/static/
\end{lstlisting}
to control the seven segement display.
\end{enumerate}
