\section{Measuring Unknown Resistance}
This section describes how to measure an unknown resistance through Vaman-ESP32
and display it on an LCD.
\subsection{Components}
The components required are listed in \autoref{table:components}.
\begin{table}[!ht]
\centering
\input{vaman-esp32/lcd/figs/components.tex}
\caption{Components}
\label{table:components}
\end{table}

\subsection{Setting up the Display}
\begin{enumerate}
\item
Plug the LCD in \autoref{fig:lcd} to the breadboard.

\begin{figure}
\centering
\includegraphics[width=\columnwidth]{vaman-esp32/lcd/figs/lcd.eps}
\caption{LCD pins}
\label{fig:lcd}
\end{figure}
\item
Connect the Vaman-ESP pins to LCD pins as per \autoref{Table:1}. Make sure that
all 5V sources are connected to the LCD through a 220 $\Omega$ resistance.
\item
The Vaman pin diagram is available in \autoref{fig:vaman-pin_sheet}.
\begin{table}[!ht]
\centering
\input{vaman-esp32/lcd/figs/table1.tex}
\caption{Make sure that all 5V sources are connected to the LCD through a 220 $\Omega$ resistance.}
\label{Table:1}
\end{table}

\item Execute the following code after editing the wifi credentials
\begin{lstlisting}
vaman-esp/lcd/codes/setup
\end{lstlisting}
You should see the following message
\begin{lstlisting}
Hi
This is CSP Lab
\end{lstlisting}
\item Modify the above code to display your name.
\end{enumerate}
\subsection{Measuring the resistance}
\begin{enumerate}

\item Connect the 5V pin of the Vaman-ESP32 to an extreme pin of the Breadboard
shown in \autoref{fig:breadboard}. Let this pin be $V_{cc}$.

\begin{figure}[!ht]
\centering
\includegraphics[width=\columnwidth]{vaman-esp32/lcd/figs/breadboard.eps}
\caption{Breadboard}
\label{fig:breadboard}
\end{figure}
%
\item
Connect the GND pin of the Vaman-ESP to the opposite extreme pin of the
Breadboard.

%
\item
Let $R_1$ be the known resistor and $R_2$ be the unknown resistor.  Connect
$R_1$ and $R_2$ in series such that $R_1$ is connected to $V_{cc}$ and $R_2$ is
connected to GND. Refer to \autoref{fig:voltage_divider}.

%
\begin{figure}[!ht]
\centering
\resizebox {\columnwidth} {!} {
\input{vaman-esp32/lcd/figs/vrr.tex}
}
\caption{Voltage Divider}
\label{fig:voltage_divider}
\end{figure}
%
\item
Connect the junction between the two resistors to  the GPIO34 pin on the
Vaman-ESP32.

%
\item
Connect the Vaman-ESP to the computer so that it is powered.
\item
Execute the following code after editing the wifi credentials

\begin{lstlisting}
vaman/vaman-esp/lcd/codes/resistance
\end{lstlisting}

\end{enumerate}
\subsection{Displaying the Measured resistance on LCD and website}
\begin{enumerate}
\item The unknown resistance is measured and diplayed the measured resistance on
the LCD display and also on the Vaman-ESP webserver.
\item Connect the Vaman-ESP pins to LCD pins as per \autoref{Table:1}.
\item
Execute the following code after editing the wifi credentials

\begin{lstlisting}
vaman-esp/lcd/webserver/codes
\end{lstlisting}
\item After flashing the code to vaman-ESP, the board will be connected to the
wifi credentials provided.
\item Now connect the same WiFi credentials to the mobile phone for accessing
the IP address, which can be accessed by 
\begin{lstlisting}
ifconfig
nmap -sn 192.168.x.x/24
\end{lstlisting}
\item Change the IP address in the second command accordingly with the IP
address provided by first command.
\item By the above commands the IP address of vaman-ESP will be diplayed.
\item Now the vaman-ESP will be hosting a webserver
\item Inorder to access the webserver type the IP address of the vaman-ESP in
the web browser.
\item In the website loaded by the IP address of vaman-ESP the Unknown resitance
is displayed as shown in \autoref{fig:website}

\begin{figure}[!ht]
\centering
\includegraphics[width=\columnwidth]{vaman-esp32/lcd/figs/website.jpg}
\caption{Website}
\label{fig:website}
\end{figure}
\end{enumerate}

\subsection{Explanation}
\begin{enumerate}

\item We create a variable called analogPin and assign it to 0. This is because 
the voltage value we are going to read is connected to analogPin GPIO34.

\item  The 12-bit ADC can differentiate 4096 discrete voltage levels, 5 volt is 
applied to 2 resistors and the voltage sample is taken in between the resistors.
The value which we get from analogPin can be between 0 and 4095. 0 would 
represent 0 volts falls across the unknown resistor. A value of 4095 would mean 
that practically all 5 volts falls across the unknown resistor.

\item  $V_{out}$ represents the divided voltage that falls across the unknown 
resistor.

\item  The Ohm meter in this manual works on the principle of the voltage
divider shown in \autoref{fig:voltage_divider}.
\begin{align}
V_{out}&=\frac{R_1}{R_1+R_2}V_{in} \\
\Rightarrow R_2&=R_1\brak{\frac{V_{in}}{V_{out}}-1}
\end{align}
In the above, $V_{in} = 5$V, $R_1 = 220 \Omega$.
\item Repeat the exercise with another unknown resistance.
\end{enumerate}
