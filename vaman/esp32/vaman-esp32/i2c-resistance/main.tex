\section{I2C Communication between Vaman-ESP32 and Two Arduinos}
This section describes how to setup the Vaman-ESP32 as a master and two Arduinos
as a slave using I2C protocol. The two unknown resistances are measured by using
two Arduinos and sending those two resistance values to Vaman-ESP32 through I2C
and displaying the unkwnown resistances on ESP32 webserver.
\subsection{Components}
The components required are listed in \autoref{table:i2c-resistance-components}.
\begin{table}[!ht]
\centering
\input{vaman-esp32/i2c-resistance/figs/components7.tex}
\caption{Components}
\label{table:i2c-resistance-components}
\end{table}
\subsection{Setting up one Master and two slaves}
\begin{enumerate}
\item
Connect the vaman-ESP pins to Arduino pins as per \autoref{Table:1
vamanesp32-arduino}.
\begin{table}[!ht]
\centering
\input{vaman-esp32/i2c-resistance/figs/table1.tex}
\caption{}
\label{Table:1 vamanesp32-arduino}
\end{table}
\item The Vaman pin diagram is available in \autoref{fig:vaman-pin_sheet}
\item
Configure Arduino Uno as a Slave-1 using the following PlatformIO and upload it.
\begin{lstlisting}
vaman-esp/i2c-resistance/codes/I2C_Sender_Arduino1
\end{lstlisting}

\item
Configure Arduino Uno as a Slave-2 using the following PlatformIO and upload it.
\begin{lstlisting}
vaman-esp32/i2c-resistance/codes/I2C_Sender_Arduino2
\end{lstlisting}

\item
Now configure vaman-ESP as a Master using the following code and upload it.
\begin{lstlisting}
vaman-esp32/i2c-resistance/codes/I2C_Reciever_ESP32
\end{lstlisting}
\end{enumerate}
\subsection{Measuring the resistance}
\begin{enumerate}

\item
Connect the 5V pin of the Vaman-ESP32 to an extreme pin of the breadboard shown 
in \autoref{fig:breadboard}.  Let this pin be $V_{cc}$.
\item
Connect the GND pin of the Vaman-ESP32 to the opposite extreme pin of the 
Breadboard.
\item
Let $R_1$ be the known resistor of 1k ohm and $R_2$ be the unknown resistor. 
Connect $R_1$ and $R_2$ in series such that $R_1$ is connected to $V_{cc}$ and 
$R_2$ is connected to GND. Refer to \autoref{fig:voltage_divider}.
\item
Connect the junction between the two resistors to the A0 pin on the Arduino 
board 1, which measures the first unknown resistance.
\item 
Connect another junction between the two resistors to the A0 pin on the Arduino 
board 2, which measures the second unknown resistance.
\item
Now power the Vaman board
\item
Execute the following PlatformIO project after editing the WiFi credentials
\begin{lstlisting}
vaman-esp32/i2c-resistance/codes/I2C_Reciever_ESP32
\end{lstlisting}
\end{enumerate}
\subsection{Displaying the Measured resistance on website}
\begin{enumerate}
\item The two unknown resistances are measured and diplayed the measured 
resistance on the Vaman-ESP32 webserver.
\item After flashing the code to Vaman-ESP32, the board will be connected to the
wifi credentials provided.
\item Now connect the same WiFi credentials to the mobile phone for accessing 
the IP address, which can be accessed by typing the following commands.
\begin{lstlisting}
ifconfig
nmap -sn 192.168.x.x/24
\end{lstlisting}
\item Change the IP address in the second command accordingly with the IP 
address provided by first command.
\item By the above commands the IP address of Vaman-ESP32 will be diplayed.
\item Now the Vaman-ESP32 will be hosting a webserver.
\item Inorder to access the webserver type the IP address of the Vaman-ESP32 in 
the web browser.
\item In the website loaded by the IP address of Vaman-ESP32 the two unknown 
resitances are displayed as shown in \autoref{fig:website_results}.
\end{enumerate}

\begin{figure}[!ht]
\centering
\includegraphics[width=\columnwidth]{vaman-esp32/i2c-resistance/figs/website.jpeg}
\caption{Website}
\label{fig:website_results}
\end{figure}
