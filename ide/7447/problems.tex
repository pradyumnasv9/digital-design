Verify all logic using the Arduino
\begin{enumerate}[label=\arabic*.,ref=\theenumi]
%		\numberwithin{figure}{enumi}
\item Obtain the Boolean Expression for the logic circuit shown below
in \figref{fig:2013/c/6/b}.
\label{prob:2013/c/6/b}

\hfill (CBSE 2013)
	\usetikzlibrary{circuits.logic.IEC,calc}
		\begin{figure}[H]
\centering
	   \begin{circuitikz} \draw
(0,2) node[or port]  (myor1) {}
(0,0) node[and port] (myand) {}
(2,1) node[or port] (myor2) {}
(myor1.out) -- (myor2.in 1)
(myand.out) -- (myor2.in 2);

\node[left] at (myor1.in 1) {\(X\)};
\node[left] at (myor1.in 2) {\(Y\)};
\node[left] at (myor1.in 1)[ocirc] {};
\node[left] at (myand.in 2) [ocirc] {};
\node[left] at (myand.in 1) {\(Y\)};
\node[left] at (myand.in 2) {\(Z\)};
\node[right] at (myor1.out) {};
\node[right] at (myand.out) {};

\node[right] at (myor2.out) {F};
\end{circuitikz}
			\caption{}
\label{fig:2013/c/6/b}
		\end{figure}
\item Verify the Boolean Expression 
\label{prob:2013/c/6/a}
\hfill (CBSE 2013)
		\begin{align*}
%\label{eq:2013/c/6/a}
	               A+C=A+A'C+BC
		\end{align*}
\item Draw the logic circuit for the following Boolean Expression 
\hfill (CBSE 2015)
\label{prob:2015-1/c/6/b}
		\begin{align*}
%\label{eq:2015-1/c/6/b}
f(x,y,z,w) = (x'+y)z + w'
		\end{align*}
\item Verify the following
\hfill (CBSE 2015)
\label{prob:2015-1/c/6/a}
		\begin{align*}
%\label{eq:2015-1/c/6/a}
U' + V = U'V' + U'V+UV
		\end{align*}
\item Draw the logic circuit for the given Boolean Expression
\hfill (CBSE 2015)
\label{prob:2015/c/6/b}
		\begin{align*}
%\label{eq:2015/c/6/b}
(U + V')W' + Z
		\end{align*}
\item 
Verify the following using Boolean Laws
\label{prob:2015/c/6/a}
\hfill (CBSE 2015)
		\begin{align*}
%\label{eq:2015/c/6/a}
X+Y' = XY+XY'+X'Y'
		\end{align*}
\item 
\label{prob:2016/c/6/b}
Write the Boolean Expression for the result of the logic circuit as shown in Fig.  
\ref{fig:2016/c/6/b}
\hfill (CBSE 2016)
\begin{figure}
\centering
\includegraphics[width=0.75\columnwidth]{figs/cbse-2016.jpg}
\caption{}
\label{fig:2016/c/6/b}
\end{figure}
\item Draw the logic circuit of the following Boolean Expression using only NAND Gates.
\hfill (CBSE 2017)
\label{prob:2017-1/c/6/b}
		\begin{align*}
%\label{eq:2017-1/c/6/b}
 XY + YZ
		\end{align*}
\item Draw the logic circuit of the following Boolean Expression using only NOR Gates  
\hfill (CBSE 2017)
\label{prob:2017/c/6/b}
      \begin{align*}
      (A+B)(C+D)
      \end{align*}
\item Draw the logic circuit of the following Boolean Expression
\hfill (CBSE 2018)
\label{prob:2018/c/6/b}
\begin{equation*} 
(U'+V)(V'+W')
\end{equation*}
\item Derive a Canonical POS expression for a Boolean function F, represented by 
Table \ref{tab:2019/c/6/c}\hfill (CBSE 2019)
\label{prob:2019/c/6/c}
\begin{table}[H]
\centering
\begin{tabular}{|l|l|l|c|}
	\hline
	X&Y&Z&F(X,Y,Z)\\
	\hline
	0&0&0&1\\
	0&0&1&0\\
	0&1&0&1\\
	0&1&1&0\\
	1&0&0&1\\
	1&0&1&1\\
	1&1&0&0\\
	1&1&1&0\\
	\hline
\end{tabular}
\caption{}
\label{tab:2019/c/6/c}
\end{table}
\item For the logic circuit shown in \figref{fig:2000/gate/ec/2/7}, find the simplified Boolean expression for the output. 
\label{prob:2000/gate/ec/2/7}
\hfill (GATE EC 2000)
\begin{figure}[H]
    \centering
    \includegraphics[width=0.75\columnwidth]{figs/2000-gate-ec-2-7.jpg}
    \caption{}
\label{fig:2000/gate/ec/2/7}
\end{figure}
\item 
Obtain the Boolean Expression for the logic circuit shown below
in \figref{fig:1993/gate/ec/4/8}.
\label{prob:1993/gate/ec/4/8}

\hfill (GATE EC 1993)
\begin{figure}[H]
    \centering
    \resizebox{0.5\columnwidth}{!}{%
	   \begin{circuitikz} \draw
(0,2) node[nand port] (mynand1) {}
(2,3) node[nand port] (mynand2) {}
(0,0) node[nand port] (mynand) {}
(2,-1) node[nand port] (mynand3) {}
(2,1) node[or port] (myor1) {}
(4,1) node[or port,number inputs =3] (myor2) {}
(mynand1.out) -- (myor1.in 1)
(mynand.out) -- (myor1.in 2)
(mynand2.out) -- (myor2.in 1)
(mynand3.out) -- (myor2.in 3)
(myor1.out) -- (myor2.in 2);
\node[left] at (mynand1.in 1) {\(A\)};
\node[left] at (mynand1.in 2) {\(B\)};
\node[left] at (mynand2.in 1) {\(A\)};
\node[left] at (mynand2.in 2) {\(A\)};
\node[left] at (mynand3.in 1) {\(C\)};
\node[left] at (mynand3.in 2) {\(C\)};
\node[left] at (mynand1.in 1)[ocirc] {};
\node[left] at (mynand.in 2) [ocirc] {};
\node[left] at (mynand.in 1) {\(B\)};
\node[left] at (mynand.in 2) {\(C\)};
\node[right] at (mynand1.out) {};
\node[right] at (mynand.out) {};
\node[right] at (mynand2.out) {};
\node[right] at (mynand3.out) {};
\node[right] at (myor2.out) {\(Y\)};
\end{circuitikz}
	}
    \caption{}
\label{fig:1993/gate/ec/4/8}
\end{figure}
%
\item Implement Table
\ref{tab:1993/gate/ec/6/13}
using XNOR logic.
\hfill (GATE EC 1993)
\label{prob:1993/gate/ec/6/13}
\begin{table}[H]
	\centering
	\begin{tabular}{|c|c|c|}
		\hline
		\textbf{A}&\textbf{B}&\textbf{Y}\\
		\hline
		0&0&1\\
		\hline
		0&1&0\\
		\hline
		1&0&0\\
		\hline
		1&1&1\\   
		\hline 
	\end{tabular}
	\caption{}
\label{tab:1993/gate/ec/6/13}
\end{table}
\item 
\label{prob:1999-gate-ec-2-11}
For a binary half-sub-tractor having two inputs A and B, find the correct set of logical expressions for the outputs $D =A \text{ minus } B$ and $X$=borrow.
\hfill (GATE EC 1999)
%
\item 
Find $X$ in the following circuit in 
\figref{fig:2007-gate-ec-43}
\hfill (GATE EC 2007)
\label{prob:2007-gate-ec-43}
\begin{figure}[H]
\centering
	\includegraphics[width=0.75\columnwidth]{figs/2007-gate-ec-43.png}
\caption{}
\label{fig:2007-gate-ec-43}
\end{figure}
\item 
\label{prob:2007-gate-in-10}
      A logic circuit implements the boolean function $F=X'.Y+X.Y'.Z'$. It is found that the input combination $X=Y=1$ can never occur. Taking this into account, find a simplified expression for $F$. 
\hfill (GATE IN 2007)
\item 
\label{prob:2010-gate-ec-39}
Find the Boolean logic realised by the following circuit in 
\figref{fig:2010-gate-ec-39}.

\hfill (GATE EC 2010)
\begin{figure}[H]
\centering
	\includegraphics[width=0.5\columnwidth]{figs/2010-gate-ec-39.png}
\caption{}
\label{fig:2010-gate-ec-39}
\end{figure}
\item 
\label{prob:2011-gate-ec-20}
Find the logic function implemented by the circuit given below 
in 
\figref{fig:2011-gate-ec-20}.

\hfill (GATE EC 2011)
\begin{figure}[H]
\centering
	\includegraphics[width=0.75\columnwidth]{figs/2011-gate-ec-20.png}
\caption{}
\label{fig:2011-gate-ec-20}
\end{figure}
\item
\label{prob:2016/gate/in/19}
Find $F$ in the Digital Circuit given in the figure below
in \figref{fig:2016/gate/in/19}.
\hfill (GATE IN 2016)
\begin{figure}[H]
	\centering
	\resizebox{0.5\columnwidth}{!}{%
\begin{tikzpicture}
% Logic ports
\node[nand port] (a) at (2,1){};
\node[nand port] (b) at (2,4){};
\node[nand port] (c) at (4,0){};
\node[nand port] (d) at (6,3){};
% Connection
\draw (a.in 2) -| (b.in 2);
\draw (b.out) -| (d.in 1);
\draw (a.out) -|  (c.in 1);
\draw (c.out) -| (d.in 2);
\draw (d.out) -- ++(1,0) node[near end,above]{F};
\draw (b.in 1) -- ++(-1.5,0)node[left](In1){X};
\draw (b.in 2) -- ++(-1.5,0)node[left](In3){Y};
\draw (c.in 2) -- ++(-1.5,0)node[left](In3){Z};
% Jump crossing element
1
\end{tikzpicture}
	}
	\caption{}
\label{fig:2016/gate/in/19}
\end{figure}


\item 
\label{prob:2017-gate-ec-16}
Find the logic function implemented by the circuit given below 
in 
\figref{fig:2017-gate-ec-16}.

\hfill (GATE EC 2017)
\begin{figure}[H]
\centering
	\includegraphics[width=0.5\columnwidth]{figs/2017-gate-ec-16.png}
\caption{}
\label{fig:2017-gate-ec-16}
\end{figure}
\item 
\label{prob:2018-gate-ee-14}
Find the logic function implemented by the circuit given below 
in 
\figref{fig:2018-gate-ee-14}

\hfill (GATE EE 2018)
\begin{figure}[H]
\centering
	\includegraphics[width=0.5\columnwidth]{figs/2018-gate-ee-14.png}
\caption{}
\label{fig:2018-gate-ee-14}
\end{figure}
\item 
\label{prob:2018-gate-CS-4}		
Let $\oplus$ and $\odot$ denote the Exclusive OR and Exclusive NOR operations, respectively.Which one of the following is NOT CORRECT ?
%\ref{prob:2018-gate-CS-4}

\hfill (GATE CS 2018)
%\begin{samepage}
\begin{enumerate}[label=(\Alph*)]
    \item $\overline{P\oplus Q}$ = $ P \odot Q $
    \item $\overline{P} \oplus Q$ = $ P \odot Q $
    \item $\overline{P} \oplus \overline{Q}$ = $ P \oplus Q $
    \item $(P \oplus \overline{P}) \oplus Q$ = $(P \odot \overline{P}) \odot \overline{Q}$
\end{enumerate}
%\end{samepage}
\item 
\label{prob:2019-gate-ee-36}
Find the logic function implemented by the circuit given below 
in 
\figref{fig:2019-gate-ee-36}

\hfill (GATE EE 2019)
\begin{figure}[H]
\centering
	\includegraphics[width=0.75\columnwidth]{figs/2019-gate-ee-36.png}
\caption{}
\label{fig:2019-gate-ee-36}
\end{figure}

\item 
\label{prob:2018-gate-ec-31}
Find the logic function implemented by the circuit given below 
in 
\figref{fig:2018-gate-ec-31}

\hfill (GATE EC 2018)
\begin{figure}[H]
\centering
	\includegraphics[width=0.75\columnwidth]{figs/2018-gate-ec-31.png}
\caption{}
\label{fig:2018-gate-ec-31}
\end{figure}
\item A Boolean digital circuit is composed using two 4-input multiplexers $M1$ and $M2$ and one 2-input multiplexer $M3$ as shown in the 
    \figref{fig:Multiplexer}.
	 $X0$–$X7$ are the inputs of the multiplexers $M1$ and $M2$ and could be connected to either $0$ or $1$. The select lines of the multiplexers are connected to Boolean variables $A$, $B$ and $C$ as shown.
Which one of the following set of values of $(X0, X1, X2, X3, X4, X5, X6, X7)$ will realise the Boolean function 
$\overline{A} + \overline{A}\overline{C}+A\overline{B}C $ ?
\hfill(GATE CS 2023)
 \begin{enumerate}
     \item (1, 1, 0, 0, 1, 1, 1, 0)
     \item (1, 1, 0, 0, 1, 1, 0, 1)
     \item (1, 1, 0, 1, 1, 1, 0, 0)
     \item (0, 0, 1, 1, 0, 1, 1, 1)
 \end{enumerate}
%
\begin{figure}[H]
    \centering
        \includegraphics[width=0.75\columnwidth]{figs/Multiplexer.png}
    \caption{Digital Circuit}
    \label{fig:Multiplexer}
\end{figure}
\item Select the Boolean function(s) equivalent to $x + yz$, where $x,y$, and $z$ are Boolean variables, and + denotes logical OR  operation.\hfill(GATE EC 2022)
	\begin{enumerate}[label=(\Alph*)]
		\item $x + z + {xy}$
		\item ${(x + y)}{(x + z)}$
		\item $x + {xy} + {yz}$
		\item $x + {xz} + {xy}$
	\end{enumerate}
 \item Which one of the following options is CORRECT for the given circuit 
			in \figref{fig:xxxx}?
	 \hfill(GATE PHYSICS 2023)
		\begin{figure}[H]
			\centering
			\includegraphics[width=0.5\columnwidth]{figs/Q24.jpg}
			\caption{}
			\label{fig:xxxx}
		\end{figure}

		\begin{enumerate}[label=(\Alph*)]
		\item P = $1$, Q = $1$ ; X = $0$
		\item P = $1$, Q = $0$ ; X = $1$
		\item P = $0$, Q = $1$ ; X = $0$
		\item P = $0$, Q = $0$ ; X = $1$
	\end{enumerate}
\item 
Consider a Boolean gate $D$ where the output $Y$ is related to the inputs $A$ and $B$ as $Y = A + B$, where + denotes logical OR operation. The Boolean inputs 0 and 1 are also available separately. Using instances of only D gates and inputs 0 and 1, select the correct option(s).\hfill{(GATE EC 2022)}
\begin{enumerate}
\item  NAND logic can be implemented
\item  OR logic cannot be implemented
\item  NOR logic can be implemented
\item  AND logic cannot be implemented.
\end{enumerate}
%
\item Let $R1$ and $R2$ be two $4$-bit registers that store numbers in $2$’s complement form.
For the operation $R1+R2$, which one of the following values of $R1$ and $R2$ gives an
arithmetic overflow?
\hfill{(GATE CS 2022)}
%
    \begin{enumerate}
        \item $R1 = 1011$ and $R2 = 1110$
        \item $R1 = 1100$ and $R2 = 1010$
        \item $R1 = 0011$ and $R2 = 0100$
        \item $R1 = 1001$ and $R2 = 1111$
    \end{enumerate}
\item The logic block shown 
in
\figref{fig:GATE IN 2021}
	has an output $F$ given by \rule{1cm}{0.5pt}.
\begin{enumerate}
	\item$A+B$
	\item$A\bar{B}$
	\item$A+\bar{B}$
	\item$\bar{B}$
\end{enumerate}
\hfill (GATE IN 2021)
\begin{figure}[H]
\centering
\includegraphics[width=0.5\columnwidth]{figs/gatemage.jpg}
	\caption{}
\label{fig:GATE IN 2021}
\end{figure}
%
\item Consider the following Boolean expression 
\begin{align*} F = (X+Y+Z)(\bar{X}+Y)(\bar{Y}+Z) \end{align*}
%       
Which of the following Boolean expressions is/are equivalent to $\overline{F}$ ?
% 
\begin{enumerate}                                     
\item $(\bar{X}+\bar{Y}+\bar{Z})(X+\bar{Y})(Y+\bar{Z})$
\item $X\bar{Y}+\bar{Z}$
\item $(X+\bar{Z})(\bar{Y}+\bar{Z})$
\item $X\bar{Y}+Y\bar{Z}+\bar{X}\bar{Y}\bar{Z}$ 
\end{enumerate}
\hfill{(GATE CS 2021)}
%
    \item The circut shown in  
\figref{fig:block_diagram}
	    comprises of XOR, AND gates and multiplexers.  
    If all the inputs $P, Q, R, S$ and $T$ are applied simultaneously and held constant, find $Y$.

\hfill(GATE EC 2021)  
\begin{figure}[H]
	\centering
	\resizebox{0.75\columnwidth}{!}{
\input{figs/figsh.tex}
	}
\caption{} 
\label{fig:block_diagram}
\end{figure}
\item  The following combination of logic gates
in
		      \figref{fig:GATE PH 2021}
	represents the operation
	\begin{enumerate}
       \item OR
       \item NAND
       \item AND
       \item NOR
   \end{enumerate}
%
 \hfill(GATE PH 2021)
	      \begin{figure}[H]
		      \centering
		      \resizebox{0.25\columnwidth}{!}{%
		      \input{figs/nand.tex}
		      }
	              \caption{}
		      \label{fig:GATE PH 2021}
	      \end{figure}
%
\item Consider the boolean Function $z\brak{a,b,c}$ from 
			\figref{fig:203} below.
		Which of the following minterm lists represent the circuit given above?
	\begin{enumerate}
		\item $z=\Sigma\brak{0,1,3,7}$
		\item $z=\Sigma\brak{1,4,5,6,7}$
		\item $z=\Sigma\brak{2,4,5,6,7}$
		\item $z=\Sigma\brak{2,3,5}$
	\end{enumerate}	   
	\hfill{(GATE CS 2020)}
%	
		\begin{figure}[H]
			\centering
			\includegraphics[width=0.5\columnwidth]{figs/203.png}
			\caption{}
			\label{fig:203}
		\end{figure}
%
\item Consider three $4$-variable functions $f_1, f_2, $and $f_3,$ which are expressed in sum-of-minterms as 
$$f_1 = \sum\brak{0,2,5,8,14}, f_2=\sum\brak{2,3,6,8,14,15}, f_3 = \sum\brak{2,7,11,14}$$. For the following circuit 
	in 
	\figref{fig:GATE-CS2019,30},
	with one AND gate and one XOR gate, the output function $f$ can be expressed as
		\begin{enumerate}
		\item $\sum\brak{7,8,11}$
		\item $\sum\brak{2,7,8,11,14}$
		\item $\sum\brak{2,14}$
		\item $\sum\brak{0,2,3,5,6,7,8,11,14,15}$
		\end{enumerate}
%
	\hfill(GATE-CS2019,30)
	\begin{figure}[H]
		 \centering
		 \resizebox{0.5\columnwidth}{!}{%
			\input{ide/7447/figs/fig2.tex}
			}
                 \caption{}
	\label{fig:GATE-CS2019,30}
	\end{figure}
\item In the circuit shown
	in \figref{fig:GATE-EC2019,14},
	what are the values of $F$ for $EN=0$ and $EN=1$,  respectively?
\begin{enumerate}
    \item $0$ and $D$
    \item $Hi-Z$ and $D$
    \item $0$ and $1$
    \item $Hi-Z$ and $\overline{D}$
\end{enumerate}
 \hfill(GATE EC 2019)  
%
\begin{figure}[H]
    \centering
    \resizebox{0.5\columnwidth}{!}{%
    \input{ide/7447/figs/fig3.tex}
	}
    \caption{Circuit Diagram}
	\label{fig:GATE-EC2019,14} 
\end{figure}
\item In the circuit shown below in 
	    \figref{fig:GATE-IN2019,34},
	 assume that the comparators are ideal and all components have zero propagation delay. In one period of the input signal $V_{in}=6\sin\brak{\omega t}$, the fraction of the time for which the output OUT is in logic HIGH is 
\begin{enumerate}[itemsep=1ex]
	\item $\frac{1}{12}$
	\item $\frac{1}{2}$
	\item $\frac{2}{3}$
	\item $\frac{5}{6}$
\end{enumerate}
		                 \hfill(GATE IN 2019)
\begin{figure}[H]
\centering
\resizebox{0.75\columnwidth}{!}{%
    \input{ ide/7447/figs/fig8.tex}
	}
	    \caption{Circuit Daigram}
	    \label{fig:GATE-IN2019,34}
     \end{figure}


\item 
	\figref{fig:GATE-IN2019,22}
	shows the $ith$ full-adder block of a binary adder circuit. $C_i$ is the input carry and $C_{i+1}$is the output carry of the circuit.  If the inputs $A_i, B_i$; are available and stable throughout the carry propagation, find the outputs $S_i$ and $C_{i+1}$.

	               \hfill(GATE IN 2019)
\begin{figure}[H] 
    \centering
    \resizebox{0.75\columnwidth}{!}{%
	\input{ide/7447/figs/fig9.tex}
	}
	\caption{Full Adder}
	\label{fig:GATE-IN2019,22}
\end{figure}
\item The Boolean operation performed by the following  circuit 
in
\figref{fig:figure13}
	at the output $O$ is \rule{1cm}{0.1pt}.
%
\begin{enumerate}

            \item  $O=S_1\oplus S_0$ 
            
            \item  $O=S_1\overline{\rm S_0}$
            
            \item  $O=S_1 + S_0$
            
            \item $O=S_0\overline{\rm S _1}$
 \end{enumerate}
    \hfill (GATE IN 2020)
%
\begin{figure}[H]
	\centering
	\resizebox{0.75\columnwidth}{!}{%
\input{ide/7447/figs/fig13.tex}
	}
\caption{}
\label{fig:figure13}
\end{figure}
\item  The chip select logic for a certain DRAM chip in a memory system design is shown below
	in
\figref{fig:figure14}.
	Assume that the memory system has 16 address lines denoted by ${A_{15}}$ to ${A_0}$. What is the range of addresses (in hexadecimal) of the memory system that can get enabled by the chip select (CS) signal?
\begin{enumerate}
\item ${C800}$ to ${CFFF}$
\item ${CA00}$ to ${CAFF}$
\item ${CA00}$ to ${C8FF}$
\item ${DA00}$ to ${DFFF}$
\end{enumerate}  
\hfill (GATE CS 2019)
%
\begin{figure}[H]
\input{ide/7447/figs/fig14.tex}
\caption{}
\label{fig:figure14}
\end{figure}
%
\item  A $2\times2$ ROM array is built with the help of diodes as shown in the circuit below
	in \figref{fig:2rom}. Here $W0$ and $W1$ are signals that select the word lines and $B0$ and $B1$ are signals that are output of the sense amps based on the stored data corresponding to the bit lines during the read operation.
%
\begin{figure}[H]
        \centering
	\resizebox{0.75\columnwidth}{!}{%
        \input{ide/7447/figs/37.tex}
	}
        \caption{ $2\times 2$ ROM array}
	\label{fig:2rom}
\end{figure}
%
		During the read operation, the selected word line goes high and the other word line is in a high impedance state. As per the implementation shown in the circuit diagram above, what are the bits corresponding to $D_{ij}\brak{\text{where $i=0$ or $1$ and $j=0$ or $1$}}$ stored in the ROM?
	\hfill(GATE EC2018,32)
\begin{enumerate}
    \item \myvec{1 & 0\\0 & 1}
    \item \myvec{0 & 1\\1 & 0}
    \item \myvec{1 & 0\\1 & 0}    
    \item \myvec{1 & 1\\0 & 0}
\end{enumerate}
%
\item $A$ and $B$ are logical inputs and $X$ is the logical output shown in 
\figref{fig:gate_in_2017_30}.
	The output $X$ is related to $A$ and $B$ by 
\begin{enumerate}
\item $X = \overline{A}B + \overline{B}A$
\item $X = AB + \overline{B}A$
\item $X = AB + \overline{B}\overline{A}$
\item $X = \overline{A}\overline{B} + \overline{B}A$
\end{enumerate}
\hfill (GATE IN 2017)
\begin{figure}[H]
\centering
\resizebox{0.75\columnwidth}{!}{%
\input{ide/7447/figs/gate_in_2017_30.tex}
	}
\caption{}
\label{fig:gate_in_2017_30}
\end{figure}
\item The functionality implemented by the circuit below 
in
\figref{fig:gate_ec_2016_43}
	is 
\begin{enumerate}
\item 2-to-1 multiplexer
\item 4-to-1 multiplexer
\item 7-to-1 multiplexer
\item 6-to-1 multiplexer
\end{enumerate}
\hfill (GATE 2016 EC)
%
\begin{figure}[H]
\centering
\resizebox{0.5\columnwidth}{!}{%
\input{ide/7447/figs/gate_ec_2016_43.tex}
	}
\caption{Multiplexer}
\label{fig:gate_ec_2016_43}
\end{figure}
%
\item A $2$-bit flash Analog to Digital Converter (ADC) is given in \figref{EE2016_37_fig1}. The input is $0 \leq V_{IN} \leq 3$ Volts. The expression of the LSB of the output $B_0$ as a boolean function of $X_2,X_1,$ and $X_0$ is 
\begin{enumerate}
\item $X_0 \left[ \overline {X_2 \oplus X_1} \right]$
\item $\overline {X_0} \left[ \overline {X_2 \oplus X_1} \right]$
\item $X_0 \left[ X_2 \oplus X_1 \right]$
\item $\overline{X_0} \left[ X_2 \oplus X_1 \right]$
\end{enumerate}
\hfill(GATE EE 2016)
\begin{figure}[H]
\centering
\resizebox{0.5\columnwidth}{!}{%
\input{ide/7447/figs/EE2016_37_fig.tex}
	}
\caption{}
\label{EE2016_37_fig1}
\end{figure}
%
\item Consider the Boolean function Z\brak{a,b,c}. Which one of the following minterm lists represents the 
 circuit given below  
 in
	 \figref{fig:GATE CS 2020}?
 \begin{figure}[H]
	 \centering
	 \resizebox{0.75\columnwidth}{!}{%
\input{ ide/7447/figs/08.tex}
	 }
	 \caption{}
	 \label{fig:GATE CS 2020}
 \end{figure}
\begin{enumerate}[label=\Alph*.]
 \item $z=\sum{\brak{0,1,3,7}}$
 \item $z=\sum{\brak{1,4,5,6,7}}$
 \item $z=\sum{\brak{2,4,5,6,7}}$
 \item $z=\sum{\brak{2,3,5}}$
\end{enumerate}
\hfill (GATE CS 2020)


\item The logic gates shown in the digital circuit below 
in
\figref{fig:GATE-EE 2018,47}
	use strong pull-down nMOS transistors for LOW logic level at the outputs. When the pull-downs are off, high -value resistors set the output logic levels to HIGH (i.e. the pull-ups are weak). Note that some nodes are intentionally shorted to implement \lq\lq wierd logic\rq\rq. Such shorted nodes will be HIGH only if the outputs of all the gates whose outputs are shorted are HIGH.
\begin{figure}[H]
\centering
\resizebox{0.75\columnwidth}{!}{%
\input{ide/7447/figs/gate18.tex}
	}
	\caption{}
\label{fig:GATE-EE 2018,47}
\end{figure}
The number of distinct values of $X_3$$X_2$$X_1$$X_0$ (out of the $16$ possible values) that give Y=1 is -----.
\hfill(GATE-EC 2018, 47)


\item In the logic circuit shown in  
\figref{fig:GATE-EE 2018,14},
	$y$ is given by
\hfill(GATE-EE 2018,14)
\begin{figure}[H]
\centering
\resizebox{0.75\columnwidth}{!}{%
\input{ide/7447/figs/gateEE18.tex}
	}
	\caption{}
\label{fig:GATE-EE 2018,14}
\end{figure}
\begin{enumerate}
    \item $Y = ABCD$
    \item $Y = ( A + B)(C + D) $
    \item $Y = A +B +C+ D$
    \item $Y = AB+CD $
    
\end{enumerate}

\item In the circuit shown in 
\figref{fig:GATE-EC 2014,15},
 if $C = 0$, the expression for $Y$ is

\hfill (GATE-EC 2014,15)
\begin{figure}[H]
\centering
\resizebox{0.75\columnwidth}{!}{%
\input{ide/7447/figs/fig16.tex}
	}
	\caption{}
\label{fig:GATE-EC 2014,15}
\end{figure}
\begin{enumerate}
    \item $Y= A\overline{B}+ \overline{A}B $ 
    \item $Y=A+B$
    \item $Y=\overline{A}+\overline{B}$
    \item $Y=AB$
\end{enumerate}
\item The logic function $f\brak{X,Y}$ realised by the given circuit in
\figref{fig:GATE-EC 2018,8}
is
\hfill{GATE-EC 2018,8}
\begin{figure}[H]
\centering
\resizebox{0.75\columnwidth}{!}{%
\input{ide/7447/figs/fig17.tex}
	}
	\caption{}
\label{fig:GATE-EC 2018,8}
\end{figure}
\begin{enumerate}
    \item NOR
    \item AND
    \item NAND
    \item XOR
\end{enumerate}

 
\item For the fallowing circuit
\figref{fig:PH2019,36},
 the correct logic values for the entries $X2$ and $Y2$ in the truth table 
in
\tabref{tab:PH2019,36}
 are
\hfill(PH2019,36)
%
\begin{figure}[H]
        \centering
	\resizebox{0.75\columnwidth}{!}{%
        \input{ide/7447/figs/logic32.tex}
	}
	\caption{}
\label{fig:PH2019,36}
       \end{figure}

		\begin{table}[H]
			\centering
			\resizebox{0.75\columnwidth}{!}{%
			\input{ide/7447/figs/table32.tex}
			}
	\caption{}
\label{tab:PH2019,36}
		\end{table}



   \begin{enumerate}
\item $1$ and $0$
\item $0$ and $0$
\item $0$ and $1$
\item $1$ and $1$

\end{enumerate}
\item 
Which one the following is not a valid identity?
\begin{enumerate}
 \item $ (x\oplus y)\oplus z = x\oplus (y\oplus z)$
 \item $ (x + y)\oplus z = x\oplus (y + z)$
 \item $ x\oplus y = x + y, if xy = 0$
 \item $ x\oplus y = (xy + x'y')'$
\end{enumerate}
\hfill{(GATE CS 2019)}
        \item Let $p$ and $q$ be two propositions. Consider the following two formulae in propositional logic.
			\begin{align*}
				 S_1 : ( \rightharpoondown p \vee (p \wedge q))\rightarrow q \\
				 S_2 : q\rightarrow(\rightharpoondown p \vee (p \wedge q))
			\end{align*}
        Which one of the following choices is correct?
		                                          \hfill(GATE-CS2021)
		\begin{enumerate}
			\item Both $S_1$ and $S_2$ are tautologies.
			\item $S_1$ is a tautology but $S_2$ is not a tautology.
			\item $S_1$ is not a tautology but $S_2$ is a tautology.
			\item Neither $S_1$ nor $S_2$ is a tautology.
		\end{enumerate}

\item P, Q, and R are the decimal integers corresponding to the  $4$-bit binary number  $1100$ consider in single magnitude, $1$'s complement, and $2$'s complement representations, respectively. The $6$-bit $2$'s complement representation of $\brak{P + Q + R}$ is

   \hfill(GATE EC-2020,38)
\begin{enumerate}
  \item $110101$
  \item $110010$
  \item $111101$
  \item $111001$
\end{enumerate}

\input{gate/EC_2015/EC2015_36.tex}
\input{gate/EC_2015/EC2015_48.tex}
\end{enumerate}


