\begin{enumerate}[label=\arabic*.,ref=\theenumi]
\item 	The state diagram of a sequence detector is shown in
  \figref{fig:gate/ec/2020/39/1}.
		 State $S_0$ is the initial state of the sequence detector. If the output is 1, then
\hfill (GATE EC 2020)
\begin{enumerate}
 \item the sequence 01010 is detected
 \item the sequence 01011 is detected
 \item the sequence 01110 is detected
 \item the sequence 01001 is detected	 
\end{enumerate}	
	\begin{figure}[H]
    \centering
    \resizebox{0.75\columnwidth}{!}{%
  \input{gate/ec/2020/39/figs/diagram.tex}
		}
  \caption{}
  \label{fig:gate/ec/2020/39/1}		
  \end{figure}	 
\item A sequence detector is designed to detect precisely 3 digital inputs, with overlapping sequences detectable. For the sequence $(1,0,1)$ and input data $(1,1,0,1,0,0,1,1,0,1,0,1,1,0)$, what is the output of this detector?
		\begin{enumerate}
			\item 1,1,0,0,0,0,1,1,0,1,0,0
			\item 0,1,0,0,0,0,0,1,0,1,0,0
			\item 0,1,0,0,0,0,0,1,0,1,1,0
			\item 0,1,0,0,0,0,0,0,1,0,0,0
		\end{enumerate}
		\hfill (GATE EE 2020)
\item Consider a $3$-bit counter, designed using T flip-flops, as shown below
in \figref{fig:3bitcounter.jpg}.
Assuming the initial state of the counter given by $PQR$ as $000$,what are the next three states?
                 \hfill(GATE CS 2021)
\begin{enumerate}
\item $011, 101, 000$
\item $010, 101, 000$
\item $010, 101, 000$
\item $010, 101, 000$
\end{enumerate}
     \begin{figure}[H]
\centering
\includegraphics[width=0.75\columnwidth]{ide/fsm/figs/3bitcounter.jpg}
\caption{}
\label{fig:3bitcounter.jpg}
\end{figure}
%
\item The state transition diagram for the circuit shown in 
	\figref{fig:GATE IN2019,39}
	is
                         \hfill(GATE IN 2019)
\begin{figure}[H]
\centering
    \resizebox{0.5\columnwidth}{!}{%
\input{ide/fsm/figs/fig11.tex}
	}
    \caption{}
	\label{fig:GATE IN2019,39}
\end{figure}
%
\begin{enumerate}
%% options1
\item 
	\figref{fig:GATE IN2019,39-1}
\begin{figure}[H]
\centering
    \resizebox{0.5\columnwidth}{!}{%
\input{ide/fsm/figs/opt1.tex}
	}
    \caption{}
	\label{fig:GATE IN2019,39-1}
\end{figure}
%% options2
\item 
	\figref{fig:GATE IN2019,39-2}
\begin{figure}[H]
\centering
    \resizebox{0.5\columnwidth}{!}{%
\input{ide/fsm/figs/opt2.tex}
	}
    \caption{}
	\label{fig:GATE IN2019,39-2}
\end{figure}
\item 
	\figref{fig:GATE IN2019,39-3}
\begin{figure}[H]
\centering
    \resizebox{0.5\columnwidth}{!}{%
\input{ide/fsm/figs/opt3.tex}
	}
    \caption{}
	\label{fig:GATE IN2019,39-3}
\end{figure}
\item 
	\figref{fig:GATE IN2019,39-4}
\begin{figure}[H]
\centering
    \resizebox{0.5\columnwidth}{!}{%
\input{ide/fsm/figs/opt4.tex}
	}
    \caption{}
	\label{fig:GATE IN2019,39-4}
\end{figure}
%
\end{enumerate}
%
\iffalse
\item A sequence detector is designed to detect precisely $3$ digital inputs, with overlapping sequences detectable. For the sequence $\brak{1,0,1}$ and input data $\brak{1,1,0,1,0,0,1,1,0,1,0,1,1,0}$, the output sequence is
$\hfill\brak{GATE\enspace EE2020-15}$
   \begin{enumerate}
  \item  $\brak{1,1,0,0,0,0,1,1,0,1,0,0}$
  \item $\brak{0,1,0,0,0,0,0,1,0,1,0,0}$
  \item $\brak{0,1,0,0,0,0,0,1,0,1,1,0}$
  \item $\brak{0,1,0,0,0,0,0,0,1,0,0,0}$

\end{enumerate}
\fi
%
\item A finite state machine (FSM) is implemented using the D flip-flops $A$ and $B$, and logic gates, as shown in  
\figref{fig:ide/fsm/figs/circuit}
	below. The four possible states of the FSM are $Q_AQ_B = 00, 01, 10$ and	 $11$.  
Assume that $X_{IN}$ is held at a constant logic level throughout the operation of the FSM. When the FSM is initialized to the state $Q_AQ_B = 00$ and clocked, after a few clock cycles, it starts cycling through
\hfill{(GATE EC 2017)}
\begin{enumerate}
\item all of the four possible states if $X_{in} = 1$
\item three of the four possible states if $X_{in} = 0$
\item only two of the four possible states if $X_{in} = 1$
\item only two of the four possible states if $X_{in} = 0$
\end{enumerate}
%
\begin{figure}[H]
\centering
\resizebox{0.75\columnwidth}{!}{%
	\input{ide/fsm/figs/circuit.tex}
}%
	\caption{}
\label{fig:ide/fsm/figs/circuit}
\end{figure}
\iffalse
\item For the given digital circuit
in	\figref{fig:Image},
	 $A = B = 1$. Assume that AND, OR, and NOT gates have propagation delays of $10\mathrm{ns}$,$10\mathrm{ns}$, and $5\mathrm{ns}$ respectively. All lines have zero
propagation delay. Given that $C = 1$ when the circuit is turned on, the frequency of steady-state oscillation of the output $Y$  is  \rule{1cm}{0.5pt}.
\hfill (GATE IN 2023)
\begin{figure}[H]
        \centering  
        
        \includegraphics[width=0.75\columnwidth]{figs/gate.png}
        \caption{Image}
	\label{fig:Image}
\end{figure}
%
\item In the circuit diagram shown below
in \figref{fig:GATE Digram}, the logic gates operate with a supply voltage of $1 V$. NAND and XNOR have $200$ps and $400$ps input-to-output delay, respectively.

At time $t=T.A(t)=0,B(t)=1 and Z(t)=0.$ When the inputs are changed to $A(t)=1,B(t)=0 \text{at} t=2T$, a 1 V pulse is observed at $Z$. the pulse width of the $1 V$ pulse is  ps.


\hfill{(GATE BM 2022)}

\begin{figure}[H]
\centering
\includegraphics[width=0.75\columnwidth]{figs/bm2022.png}
\caption{}
\label{fig:GATE Digram}
\end{figure}

\begin{enumerate}
\item $100$
\item $200$
\item $400$
\item $600$
\end {enumerate}
%
\fi
\item Find the states in 
	\figref{fig:GATEEC2020-50}.
\hfill (GATE EC 2020)
%
	\begin{figure}[H]
    \centering
    \resizebox{0.75\columnwidth}{!}{%
\input{ide/7474/figs/fig16.tex}
	}
    \caption{}
	\label{fig:GATEEC2020-50}
\end{figure}

\end{enumerate}
