\subsection{Arduino Droid}
For flashing the ESP32 through OTG follow the below steps:
\begin{enumerate}
	\item Install ArduinoDroid from apkpure.
	\item Open ArduinoDroid and grant all permissions
	\item Connect the ESP32 to your phone via USB-OTG and select the board \textbf{DOIT ESP32 DEVKIT V1} in ArduinoDroid using the below path.
	\begin{lstlisting}
Settings->Board type->ESP32->DOIT ESP32 DEVKIT V1
	\end{lstlisting}
    See \figref{fig:esp32}
for DOIT ESP32 DEVKIT V1.
		For ESP32 \textbf{NodeMCU} ( see
		\figref{fig:nodemcu}),
	\begin{lstlisting}
Settings->Board type->ESP32->NodeMCU-32S
	\end{lstlisting}
	\item Run the blink program present in the below link
	\begin{lstlisting}
esp32/ide/blink/src/main.cpp
	\end{lstlisting}
	The in-built led will start blinking.\\
\end{enumerate}
\subsection{Platformio}
	\begin{enumerate}
	\item In termux excecute the following to generate the bin file.
	\begin{lstlisting}
cd esp32/ide/blink
pio run
	\end{lstlisting}
\item Copy the bin file generated to ArduinoDroid/precompiled directory 
	\label{esp32:firmware}
	\begin{lstlisting}
cp .pio/build/esp32doit-devkit-v1/firmware.bin /sdcard/ArduinoDroid/precompiled/
	\end{lstlisting}
	\item Flash the bin file to ESP32 using ArduinoDroid. Open ArduinoDroid, 
	\begin{lstlisting}
Actions->Upload->Upload Precompiled->firmware.bin
	\end{lstlisting}
	After the upload is finished we get the below error in ArduinoDroid terminal. This indicates that the code is uploaded.
	\begin{lstlisting}
Error: open failed: ENOENT (No such file or directory)
	\end{lstlisting}
	Disconnect the power supply from ESP32 and reconnect it. The onboard LED should blink.
	\end{enumerate}
\subsection{OTA: Wireless Flashing}
\begin{enumerate}
	\item Setup your mobile hotspot with 
	\begin{lstlisting}
	username: npal
	password: npal1234
	\end{lstlisting}
	\item Connect the esp32 to your mobile through USB.
	\item Execute the following commands using platformio
	\begin{lstlisting}
cd esp32/ide/ota/setup
pio run
	\end{lstlisting}
	and follow the instructions from Item 
	\ref{esp32:firmware} onwards.
	\item Check the connected devices in your hotspot settings.  You should see the ESP as a connected device.  Obtain the IP address of your ESP from the hotspot settings.
		Now execute the following in termux.
	\begin{lstlisting}
cd esp32/ide/ota/blink
pio run
pio run -t nobuild -t upload --upload-port 192.168.xx.xx #replace xx with IP of ESP32
	\end{lstlisting}
\end{enumerate}

\begin{figure}[h]
    \centering
    \includegraphics[width=1\linewidth]{esp32/figs/esp32.png}
    \caption{ESP32 Devkit-v1}
    \label{fig:esp32}
\end{figure}
\begin{figure}[h]
    \centering
    \includegraphics[width=1\linewidth]{esp32/figs/nodeMCU.png}
    \caption{ESP32 NodeMCU}
    \label{fig:nodemcu}
\end{figure}
In the following, we will use the OTA method for flashing the ESP32 using platformio.
\iffalse
\par The ESP32-devkit-v1 as shown in Figure \ref{fig:esp32} has some ground pins, ADC\brak{Analog to Digital Converter} input pins D2, D4, D12-D15, D25-D27, D32 and D33 that can be used for both input as well as output. It also has two power pin that can generate 3.3$V$.  In the following exercises, only the GND, 3.3$V$ and digital pins will be used. Similarly ESP32 NodeMCU as shown in \ref{fig:nodemcu} also has ADC pins startin with P instead of D.
\fi
\subsection{Seven Segment Display}
\begin{enumerate}[label=\arabic*.,ref=\theenumi]
\item
Make connections according to Table \ref{table:ard_disp_num}
%\iffalse
\begin{table}[H]
\centering
\input{esp32/ide/sevenseg/figs/ard_disp_num}
\caption{}
\label{table:ard_disp_num}
\end{table}
%\fi

\item
Execute the following
%
\begin{lstlisting}
esp32/ide/sevenseg/src/main.cpp
\end{lstlisting}
%
The number 5 should be displayed.
\item
Now generate the numbers 0-9 by modifying the above program.

\end{enumerate}


\subsection{7447}
\begin{enumerate}
\item Make the connections as per Table \ref{table:7447_ard}  and execute the following program.  You should see the number 5 displayed.
\begin{lstlisting}
/esp32/ide/7447/codes/display/src/main.cpp
\end{lstlisting}
\begin{table}[H]
\centering
\input{esp32/ide/7447/figs/7447_ard.tex}
\caption{}
\label{table:7447_ard}
\end{table}
\item Now execute the following code.  You should see the number 2 being displayed.  This code increments the input by 1.
\begin{lstlisting}
esp32/ide/7447/codes/inc_dec/src/main.cpp
\end{lstlisting}
\item Make additional connections as shown in Table \ref{ip_7447_ard}. You should see the number 6 displayed. The code increments the manually given input to the 7447 IC by 1.
\begin{lstlisting}
esp32/ide/7447/codes/ip_inc_dec/src/main.cpp
\end{lstlisting}
\begin{table}[H]
\centering
\input{esp32/ide/7447/figs/ip_7447_ard.tex}
\caption{}
\label{table:ip_7447_ard}
\end{table}
\end{enumerate}


\subsection{7474}
We show how to use the 7474 D-Flip Flop ICs in
a sequential circuit to realize a decade counter.
\iffalse
\subsection{Components}
\begin{table}[H]
\centering
\input{ide/7474/figs/components.tex}
\caption{}
\label{table:components-7474}
\end{table}
\fi
\subsection{Decade Counter}
\begin{enumerate}[label=\arabic*.,ref=\theenumi]
\item
Generate the CLOCK signal using the \textbf{blink} program in the arduino. 
\item
Connect the Arduino, 7447 and the two 7474 ICs according to Table \ref{fig:ff_ard_pin} and Fig. \ref{fig:decade_counter}. The pin diagram for 7474 is available in Fig. \ref{fig:7474}
			\begin{table}[H]
%\begin{table}
\centering
\input{ide/7474/figs/ff_ard_pin.tex}
\caption{}
\label{fig:ff_ard_pin}
%\end{table}
\end{table}
%
\begin{figure}[H]
\begin{center}
\resizebox {0.75\columnwidth} {!} {
\input{ide/7474/figs/7474.tex}
}
\end{center}
\caption{}
\label{fig:7474}
\end{figure}

%
\item
Intelligently use the codes in 
\begin{lstlisting}
ide/7447/codes/inc_dec/inc_dec.ino
\end{lstlisting}
and
\begin{lstlisting}
ide/7447/codes/inc_dec/ip_inc_dec.ino
\end{lstlisting}
to realize the decade counter in Fig. \ref{fig:decade_counter}.
% 
 \begin{figure}[H]
\begin{center}
\resizebox {0.75\columnwidth} {!} {
\input{ide/7474/figs/decade_counter.tex}
}
\end{center}
\caption{}
\label{fig:decade_counter}
\end{figure}
%
	\end{enumerate}



%\end{document}
